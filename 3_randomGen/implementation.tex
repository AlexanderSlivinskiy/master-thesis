\chapter{Implementing randomly generated stochastic games} \label{ch:implementedRandomGen}
In this chapter, we discuss how our implementation of random generation differs from the theory described in Chapter \ref{ch:randomGen} as well as how to use and extend our implementation. 

\section{Parameters and guidelines for model construction} \label{sec:guidelines}
We have implemented a constrained version of Algorithm \ref{alg:randomRandom} from Section \ref{sec:randomGenAlgo} to randomly generate models.
The real-world implementation has to be constrained due to the natural restriction that a computer cannot generate arbitrarily big stochastic games and 
arbitrarily small transitions due to finite memory. Additional constraints on the real-world implementation are the pseudo-randomness while taking decisions as well as
floating-point machine precision.

However, usually, the users also want some control over the properties of the resulting models like the number of states.
If all model properties are differing too much, it is very hard to deduct why an algorithm performs differently on two models.
For example, it is very likely that if we would always randomize the number of states of a model, the models would differ so much in their number of unknown states that all other structural differences 
would have a comparatively diminishing impact on algorithm performance.

Thus, we provide several parameters which we do not randomize. Instead, we give the user the option of providing a value for the parameter or using our default parameters.
The parameters in Subsection \ref{list:params} provide an overview of what we do not randomize in our implementation.
Of course, it is possible to wrap a script around our generation implementation that randomizes the parameters to increase the variation of generated models.

\subsection*{Limits and additional guideline options} \label{sec:guidelinesSubsec}
Although the parameters we expose for random generation constrain some structural properties of the resulting models, 
there are many structural properties we can only barely influence. For example, there is no obvious way how to influence the size of SCCs that a model has or to guarantee
that every state in a model has a certain number of actions when using Algorithm \ref{alg:randomRandom}.

For this, we provide the option in our implementation to use other scripts for the model generation that are wrapped into PRISM files. We refer to these scripts as guidelines.
We provide two additional guidelines to our random-generation procedure described in Section \ref{sec:randomGenAlgo}: 
A procedure that controls how many actions each state has and another one that provides a guarantee on the size of SCCs in the model.
Both guidelines generate model $\SG \in \connectedSG$ but are not able to create any game in $\connectedSG$.

The guideline that controls how many actions a state has is called \emph{RandomTree} 
because it creates a treelike graph structure where the initial state is the root.
Every node of the tree has $k$ actions and at most $k$ children, where $k$ is a parameter. 
Every action has an assigned child to which it has a positive transition probability.
The rest of the probability distribution of the action is assigned at random.
An inner node of the tree may have less than $k$ children if adding $k$ children would exceed the requested number of states n for the model.
Also, leaves are not required to have $k$ actions.
Their actions are only introduced during the backward process and enable the generation of end components.

We refer to the guideline that controls the SCCs of a model by \emph{RandomSCC}.
The procedure requires a minimal and maximal size boundary $[a, b]$ for every SCC and the total number of states in the stochastic game $n$.
First, we create subgames of a randomly chosen size in $[a,b]$.
The subgames are created by the algorithm described in Section \ref{sec:randomGenAlgo}.
We then use Tarjan's algorithm for strongly connected components \cite{TarjansAlgorithm} to identify the SCCs of the created subgame.
Next, we unify all the SCCs of the subgame by using the topological enumeration Tarjan's algorithm provides.
We circularly connect the SCCs following the enumeration, making the whole subgame an SCC.
Next, we make sure that the subgame is connected to the rest of the stochastic game by making sure a previously created subgame has an action
leading into this subgame. We repeat this procedure until we have at least $n$ states in the stochastic game, 
resulting in a stochastic game $\SG \in \connectedSG$ where the user has an easy way of controlling the size of the SCC in $\SG$.

\section{A manual on how to use our implementation}
In this section, we describe the implementation details and provide guidance for using, understanding, and extending our implementation.
The random generation is split into multiple Python modules to maximize extendability and readability. 
We recall that a Python module is a file with a script.
All relevant Python modules are located in the folder "random-generated-models" in the GitHub Repository \url{https://github.com/ga67vib/Algorithms-For-Stochastic-Games}.
We provide a description of how to use the modules and how to extend them in case the reader wants to.

\subsubsection*{Generating models} \label{sec:paramExplanation}
If you want to generate a model, you can run the \texttt{modelGenerator.py}-script.
This script can receive several parameters to guide the generation process and constrain which models can be created.

You can use \texttt{python modelGenerator -help} to see all possible parameters and what they do.
Here is the exhaustive list of parameters:
\begin{itemize} \label{list:params}
    \item outputDir: Where to put the resulting model?
    \item size: How many states should the generated model have?
    \item numModels: How many models to create?
    \item minIncomingActions: The minimum of how many actions should lead into a state during the forward procedure of algorithm \ref{alg:randomRandom}.
    \item maxIncomingActions: The maximum of how many actions should lead into a state during the forward procedure of algorithm \ref{alg:randomRandom}.
    \item maxBackwardsActions: The maximum of how many actions should be added to a state during the backward procedure of algorithm \ref{alg:randomRandom}.
    \item guideline: Which guideline to use when generating the model. Will use by default Algorithm \ref{alg:randomRandom}.
    Alternatives are the "RandomTree" and "RandomSCC" guidelines that are described in Section \ref{sec:guidelinesSubsec}.
    \item smallestProb: What is the smallest probability that is allowed to occur?
    \item backwardsProb: If a random probability is generated below this threshold, the state receives no actions in the backward procedure.
    \item branchingProb: If a random probability is generated below this threshold, the state-action pair leads into only one state with probability 1.
    \item maxBranchNum: How many positive transition probabilities may one action have at most?
    \item forceUnknown: A switch. If set to true, a new target and sink are introduced that effectively multiply the value of each state by 0.9.
    This ensures that no additional targets are created during pre-computation. 
    We introduce the switch to ensure that the generated model is not solved mostly during pre-computation.
    \item maximizerProb: What is the probability that a state belongs to the Maximizer?
    \item minSCCSize: What is the smallest size an SCC may have (only for RandomSCC guideline)?
    \item maxSCCSize: What is the largest size an SCC may have (only for RandomSCC guideline)?
    \item fastTransitions: When used, keep a stack of shuffled state indices that is popped whenever a successor for a transition is needed. 
    Transition goals are more predictable but generated significantly faster.
\end{itemize}

The generated model will be named by the parameter values and will be stored in the specified directory or \texttt{generatedModels} by default.

To generate multiple models with different parameter settings, we have used the simple script \texttt{massModelGenerator.py} which calls \texttt{modelGenerator} various times with the different sets of parameters.

\subsubsection*{Module explanation}
Next, we explain the functionality of every module. The module \texttt{modelGenerator.py} is responsible for calling the module that generates the stochastic game and translating it into a .prism file. 
Furthermore, it is the interface for users.

The creation of the stochastic game is outsourced to another python module. The random generation as described in Algorithm \ref{alg:randomRandom} happens in
module \texttt{graphGenerator.py}. This is also the base class for other procedure guidelines. 
Both the RandomTree guideline - implemented in model \texttt{treeGraphGenerator.py} - and the RandomSCC guideline - implemented in module \texttt{sccGraphGenerator.py} - inherit
from \texttt{graphGenerator}.

The module \texttt{graphGenParams.py} contains a class that contains all the parameters any graph generator requires. 

Lastly, there is a module called \texttt{randomStateGetter.py}. This class implements various randomPickers.
A randomPicker selects a state of the state-space randomly.
For example, this is used to decide which state should have an action to the currently introduced state.
Different implementations allow us to guide the selection process to achieve different distributions of actions and transitions.
Each generator receives two randomPickers as part of their input parameters: One that decides which state should receive an outgoing action, 
and another one that decides where a transition should lead. By default, the randomPicker picks any state $\in [\initstate, \state<m>]$ randomly at uniform, where $m$ is a parameter.

\subsubsection*{Module extension}
If you are interested in implementing a new way of generating random models by providing new guidelines, all you have to do is create a new
module that inherits from \texttt{GraphGenerator} and implement the functionality. The \texttt{generateGraph()} call must receive a \texttt{GraphGenParams} object and return nothing.
After \texttt{generateGraph()}, the \texttt{GraphGenerator} instance should hold all the information about the game in its class variables.
\FloatBarrier