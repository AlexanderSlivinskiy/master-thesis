\begin{figure}


\centering
\begin{tikzpicture}[scale=0.6, every node/.style={transform shape}]
%little BCEC with loop
\drawdummy (init) at (-2,0) {};
\drawbox (s0) at (0,0) {$\mathsf{s_0}$};
\drawbox (s0a) at (4,0) {$\mathsf{s_{0,a}'}$};
\drawbox (s1) at (16,0) {$\target$};
\drawdummy (s0Mid) at (2,2) {};

\drawbox (m1) at (4,4) {$\mathsf{h_1}$};
\drawdummy (m1Mid) at (6,4) {};
\drawbox (m2) at (8,4) {$\mathsf{h_2}$};
\drawdummy (m2Mid) at (10,4) {};
\drawdummy (helper1) at (11,4) {};
\drawdummy (helper2) at (13,4) {};
\drawbox (m3) at (16,4) {$\mathsf{h_m}$};
\drawdummy (m3Mid) at (16,2) {};
\drawcirc (0) at (18,2)  {$\mathrm{\zeroSink}$};

\filldraw 
(11.5,4) circle (2pt)
(12,4) circle (2pt) 
(12.5,4) circle (2pt);

%\drawbox (1) at (4,4) {$\target$ };

\draw[->] (init) to (s0);
\draw[-] (s0) to node [left ,midway] {$\mathsf{a}$}(s0Mid);
\draw[->] (s0Mid) to node [text width =1.0cm, align=left, midway, below] {$\frac{1}{2}$} (s0a);
\draw[->] (s0Mid) to node [text width =1.0cm, near end, above] {$\frac{1}{2}$} (m1);
\draw[->] (s0a) to node [above, midway] {$\mathsf{a}$}(s1);

\draw[-] (m1) to node [above,midway] {$\mathsf{a}$}(m1Mid);
\draw[->] (m1Mid) to node [text width =1.0cm, midway, below] {$\frac{1}{2}$} (s0a);
\draw[->] (m1Mid) to node [text width =1.0cm, midway, above] {$\frac{1}{2}$} (m2);

\draw[-] (m2) to node [above,midway] {$\mathsf{a}$}(m2Mid);
\draw[->] (m2Mid) to node [text width =-1.0cm, midway, right] {$\frac{1}{2}$} (s0a);
\draw[-] (m2Mid) to node [text width =1.0cm, midway, above] {$\frac{1}{2}$} (helper1);
\draw[->] (helper2) to node [text width =1.0cm, midway, below] {$\frac{1}{2}$} (m3);

\draw[-] (m3) to node [left,midway] {$\mathsf{a}$}(m3Mid);
\draw[->] (m3Mid) to node [text width =1.0cm, midway, above] {$\frac{1}{2}$} (s0a);
\draw[->] (m3Mid) to node [midway, above] {$\frac{1}{2}$} (0);

\draw[->]  (s1) to [loop right] node [midway,right] {$\mathsf{a}$} (s1);
\draw[->]  (0) to [loop right] node [midway,right] {$\mathsf{a}$} (0);


\end{tikzpicture}
\caption[Example of how to transform into stopping games]{An example of adding the $\epsilon$-transitions to a simple $SG$ where initially there were only $s_0$, $\target$ and the action $\action$ with $\post(s_0, \action) = \post(\target, \action) = \target$. The action of $s_{0,a}$ is simulating the outcome of taking $\action$ in $s_0$ in the initial $SG$ while $h_1, h_2, ... h_m$ adds the $\epsilon$ needed if the initial $SG$ would be non-stopping}
\label{ex:exampleToStopping}
\end{figure}
