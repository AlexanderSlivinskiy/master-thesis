\begin{figure}

\centering
\begin{tikzpicture}[thick,scale=0.6, every node/.style={transform shape}]
%little BCEC with loop
\drawdummy (init) at (4,2) {};
\drawbox (q) at (4,0) {$\mathsf{s_0}$};
\drawdummy (mid) at (4,-2) {};
\drawbox (11) at (8,-4) {$\mathsf{f_1}$};
\drawbox (12) at (4,-4) {$\mathsf{f_2}$};
\drawcirc (0) at (0,-4)  {$\mathrm{\zeroSink}$};

\drawdummy (l) at (9, -2) {};
\drawdummy (r) at (11, -2) {};

\draw[->] (init) to (q);
\draw[-]  (q) to node [left ,midway] {$\mathsf{b}$}(mid);
\draw[->] (mid) to node [midway, above] {$\frac{1}{2}$} (0);
\draw[->] (mid) to node [midway, right] {$\frac{1}{2}$} (12);
\draw[->]  (q) to node [left ,midway] {$\mathsf{a}$}(0);
\draw[->]  (q) to node [above ,midway] {$\mathsf{c}$}(11);

\draw[->]  (0) to[loop below]  node [midway,below] {$\mathsf{a}$} (0);
\draw[->]  (11) to [loop below] node [midway,below] {$\mathsf{a}$} (11);
\draw[->]  (12) to [loop below] node [midway,below] {$\mathsf{a}$} (12);

\draw[->] (l) to (r);
\end{tikzpicture}
\begin{tikzpicture}[thick,scale=0.6, every node/.style={transform shape}]
%little BCEC with loop
\drawdummy (init) at (4,2) {};
\drawbox (q) at (4,0) {$\mathsf{s_0}$};
\drawbox (p) at (2,-4){$\mathsf{s_1'}$}; 
\drawdummy (mid) at (2,-6) {};
\drawbox (11) at (8,-8) {$\mathsf{f_1}$};
\drawbox (12) at (4,-8) {$\mathsf{f_2}$};
\drawcirc (0) at (0,-8)  {$\mathrm{\zeroSink}$};

\draw[->] (init) to (q);
\draw[-]  (p) to node [left ,midway] {$\mathsf{b}$}(mid);
\draw[->] (mid) to node [midway, above] {$\frac{1}{2}$} (0);
\draw[->] (mid) to node [midway, above] {$\frac{1}{2}$} (12);
\draw[->]  (p) to node [left ,midway] {$\mathsf{a}$}(0);
\draw[->]  (q) to node [right ,midway] {$\mathsf{b}$}(11);
\draw[->]  (q) to node [left ,midway] {$\mathsf{a}$}(p);

\draw[->]  (0) to[loop below]  node [midway,below] {$\mathsf{a}$} (0);
\draw[->]  (11) to [loop below] node [midway,below] {$\mathsf{a}$} (11);
\draw[->]  (12) to [loop below] node [midway,below] {$\mathsf{a}$} (12);


\end{tikzpicture}
\caption[Transformation into $\twoSucc$]{An example of transforming an $\SG$ into one fulfilling the $\twoSucc$-constraint. $s_0$ has more than two actions, so a binary subtree is built up where each states except for the leafs $f_1, f_2, \zeroSink$ has exactly two actions.}
\label{ex:2Succ}
\end{figure}