\begin{figure}


\centering
\begin{tikzpicture}[scale=0.8, every node/.style={transform shape}]
%little BCEC with loop
\drawdummy (init) at (-2,0) {};
\drawbox (q) at (0,0) {$\mathsf{s_0}$};
\drawcirc (p) at (0,4) {$\mathsf{s_1}$};
\drawdummy (mid) at (2,2) {};
\drawbox (1) at (4,4) {$\target$ };
\drawcirc (0) at (4,0)  {$\mathrm{\zeroSink}$};

\draw[->] (init) to (q);
\draw[->]  (q) to [bend right] node [left ,midway] {$\mathsf{a}$}(p);
\draw[->]  (p) to [bend right] node [right, midway] {$\mathsf{a}$}(q);
\draw[->]  (p) to node [midway,anchor=south] {$\mathsf{b}$}(1) ;
\draw[-] (q) to node [midway,anchor=south] {$\mathsf{b}$} (mid) ;
\draw[->] (mid) to node [text width =1.0cm, midway, below] {$\frac{1}{2}$} (0);
\draw[->] (mid) to node [text width =1.0cm, near end, above] {$\frac{1}{2}$} (1);
%\draw (mid) to [bend left,out=270,in=270] (q) ;
\draw[->]  (0) to[loop right]  node [midway,anchor=south] {$\mathsf{a}$} (0);
\draw[->]  (1) to [loop right] node [midway,anchor=south] {$\mathsf{a}$} (1);


\end{tikzpicture}
\caption[Example of an $\SG$ with an end component of size 2]{An example of a non-stopping $\SG$ with an end component of size 2. If in $s_0$ action $a$ and in $s_1$ action $a$ are chosen for some strategies $\straa, \strab$ the game does never stop. $\valstra (s_0)=0$ because the probability to reach any $f \in \fstates$ is 0 with these strategies.}
\label{ex:exampleEC}
\end{figure}
