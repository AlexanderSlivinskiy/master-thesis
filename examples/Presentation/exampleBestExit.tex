\begin{figure}


\centering
\begin{tikzpicture}[scale=0.55, every node/.style={font=\LARGE, transform shape}]
%little BCEC with loop
\drawdummy (init) at (-2,0) {};
\drawbox (q) at (0,0) {$\mathsf{s_0}$};
\drawbox (p) at (4,0) {$\mathsf{s_1}$};
\drawdummy (mid) at (6,0) {};
\drawbox (1) at (8,1) {$\target$ };
\drawcirc (0) at (8,-1)  {$\mathrm{\zeroSink}$};

\drawdummy (l) at (9, 0) {};
\drawdummy (r) at (11, 0) {};

\draw[->] (init) to (q);
\draw[->]  (q) to [bend right] node [below ,midway] {$\mathsf{a}$}(p);
\draw[->]  (p) to [bend right] node [above, midway] {$\mathsf{a}$}(q);
\draw[-]  (p) to node [midway,anchor=south] {$\mathsf{b}$}(mid) ;
\draw[->] (mid) to node [midway, below] {$\frac{1}{3}$} (0);
\draw[->] (mid) to node [midway, above] {$\frac{1}{3}$} (1);
\draw[->] (mid) to [bend right,out=270, in=270] node [midway, above] {$\frac{1}{3}$} (q);


%\draw (mid) to [bend left,out=270,in=270] (q) ;
\draw[->]  (0) to[loop right]  node [midway, right] {$\mathsf{a}$} (0);
\draw[->]  (1) to [loop right] node [midway, right] {$\mathsf{a}$} (1);

\draw[->] (l) to (r);
\end{tikzpicture}
\begin{tikzpicture}[scale=0.55, every node/.style={font=\LARGE, transform shape}]
%little BCEC with loop
\drawdummy (init) at (-2,0) {};
\drawbox (q) at (0,0) {$\mathsf{s_0}$};
\drawbox (p) at (4,0) {$\mathsf{s_1}$};
\drawdummy (mid) at (6,0) {};
\drawbox (1) at (8,1) {$\target$ };
\drawcirc (0) at (8,-1)  {$\mathrm{\zeroSink}$};

\draw[->] (init) to (q);
\draw[->]  (q) to [bend right] node [below ,midway] {$\mathsf{a}$}(p);
\draw[->]  (p) to [bend right] node [above, midway] {$\mathsf{a}$}(q);
\draw[-]  (p) to node [midway,anchor=south] {$\mathsf{b}$}(mid) ;
\draw[->] (mid) to node [midway, below] {$\frac{1}{2}$} (0);
\draw[->] (mid) to node [midway, above] {$\frac{1}{2}$} (1);


%\draw (mid) to [bend left,out=270,in=270] (q) ;
\draw[->]  (0) to[loop right]  node [midway, right] {$\mathsf{a}$} (0);
\draw[->]  (1) to [loop right] node [midway, right] {$\mathsf{a}$} (1);
\end{tikzpicture}
\caption[Example of a redistribution of probabilities for bestExit]{An example of a MEC $T = \{s_0, s_1\}$ where $s_0, s_1 \in \states<\Box>$. ($\state, \actionb$) leads to exits $\zeroSink$, $\target$, but also back to $s_0$. However, as the Maximizer can always get from $s_0$ back to $s_1$ this transition is not relevant for $\val(s_1)$. We adjust the probabilities of ($\state, \actionb$) and get an equivalent $\SG$, where $\val(s_1)$ depends only on $\target$ and $\zeroSink$ without changing the value of any state.}
\label{ex:exampleBestExit}
\end{figure}
