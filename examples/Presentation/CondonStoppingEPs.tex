\documentclass{standalone}
\usepackage{tikz}

% == Tikz
\newcommand{\drawcirc}{\node[draw,circle,minimum size=1.5cm]}
\newcommand{\drawcircred}{\node[red, draw,circle,minimum size=1.5cm]}
\newcommand{\drawbox}{\node[draw,rectangle,minimum size=1.5cm]}
\newcommand{\drawboxred}{\node[red, draw,rectangle,minimum size=1.5cm]}
\newcommand{\drawdummy}{\node[minimum size=0,inner sep=0]}
\newcommand{\zeroSink}{\mathsf{o}}
\newcommand{\target}{\mathsf{f}}

\begin{document}
\begin{tikzpicture}[scale=0.6, every node/.style={transform shape}]
%little BCEC with loop
\drawdummy (init) at (-2,0) {};
\drawdummy (space1) at (0,3.5) {};
\drawdummy (space2) at (0,-3.5) {};
\drawbox (s0) at (0,0) {$\mathsf{s_0}$};
\drawdummy (s0Mid) at (2,0) {};
\drawbox (s1) at (4,-2) {$\target$};
\drawcirc (0) at (4,2)  {$\mathrm{\zeroSink}$};

%\drawbox (1) at (4,4) {$\target$ };

\draw[->] (init) to (s0);
\draw[-] (s0) to node [midway,above] {$\mathsf{a}$} (s0Mid);
\draw[->] (s0Mid) to node [text width=0.5cm, midway, above] {$\epsilon$} (0);
\draw[->] (s0Mid) to node [text width=1.5cm, midway, below] {$1-\epsilon$} (s1);

\draw[->]  (s1) to [loop right] node [midway,right] {$\mathsf{a}$} (s1);
\draw[->]  (0) to [loop right] node [midway,right] {$\mathsf{a}$} (0);
\end{tikzpicture}
\end{document}
