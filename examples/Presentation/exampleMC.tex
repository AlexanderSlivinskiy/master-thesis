\begin{figure}


\centering
\begin{tikzpicture}[scale=0.6, every node/.style={font=\LARGE, transform shape}]
%little BCEC with loop
\drawdummy (init) at (-2,0) {};
\drawbox (m0) at (0,0) {$\mathsf{m_0}$};
\drawcirc (m1) at (-2,5) {$\mathsf{m_1}$};
\drawcirc (m2) at (2,4) {$\mathsf{m_2}$};
\drawdummy (mid) at (2,2) {};
\drawdummy (mid2) at (3,5) {};
\drawbox (s1) at (4,0) {$s_3$};
\drawbox (s2) at (6,4) {$s_4$};
\drawbox (1) at (8,2)  {$\target$};

\draw[->] (init) to (m0);
\draw[-]  (m0) to (mid);
\draw[->] (mid) to node [midway, left] {$\frac{1}{2}$} (m2);
\draw[->] (mid) to node [text width=-0.75cm, midway, right] {$\frac{1}{2}$} (s1);
\draw[->]  (m1) to (m2);
\draw[-]  (m2) to (mid2);
\draw[->] (mid2) to node [midway, above] {$\frac{3}{4}$} (m1);
\draw[->] (mid2) to node [midway, above] {$\frac{1}{4}$} (s2);
\draw[->]  (s1) to (1);
\draw[->]  (s2) to (1);
\draw[->]  (1) to [loop right] (1);


\end{tikzpicture}
\caption[Example of an induced Markov chain]{An example of an induced Markov chain $\G[\straa ,\strab]$ to some $\SG$. 
Let the vertices $\{m_0, m_1, m_2\}$ be a MEC in the original $\SG$. In $\G[\straa,\strab]$ the values of the states in the MEC can be rephrased depending on the values of the exits of the MEC, e.g. $\valstra (m_0) = p_{m_0}^{\straa ,\strab} (s_3) \cdot \valstra (s_3) + p_{m_0}^{\straa,\strab} (s_4) \cdot \valstra (s_4)$.}
\label{ex:exampleMC}
\end{figure}
