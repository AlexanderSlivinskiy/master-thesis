\begin{figure}


\centering
\begin{tikzpicture}
%little BCEC with loop
\drawdummy (init) at (-2,0) {};
\drawcirc (p) at (0,0) {$\mathsf{p}$};
\drawbox (q) at (4,0) {$\mathsf{q}$};
\drawdummy (mid) at (6,0) {};
\drawbox (1) at (8,2) {$\target$ };
\drawcirc (0) at (8,-2)  {$\bot$};

\draw[->] (init) to (p);
\draw[->]  (p) to[bend left] node [midway,anchor=south] {$\mathsf{a}$}(q) ;
\draw[->]  (q) to [bend left] node [midway,anchor=north] {$\mathsf{b}$} (p);
\draw[-] (q) to node [midway,anchor=south] {$\mathsf{c}$} (mid) ;
\draw (mid) -- (0);% node [midway,anchor=north east] {0.5};
\draw (mid) -- (1);%node [midway,anchor=south east] {0.5};
\draw (mid) to [bend left,out=270,in=270] (q) ;
\draw[->]  (0) to[loop above]  node [midway,anchor=south] {$\mathsf{e}$} (0);
\draw[->]  (1) to [loop above] node [midway,anchor=south] {$\mathsf{d}$} (1) ;


\end{tikzpicture}
\caption{An extension of the example $\SG$ in Figure \ref{ex:littleBCEC}, where the action $\mathsf{c}$ has been changed. In this example $\trans(\mathsf{q},\mathsf{c})(\mathsf{q}) = \trans(\mathsf{q},\mathsf{c})(\target) = \trans(\mathsf{q},\mathsf{c})(\bot) = \frac{1}{3}$.}
\label{ex:littleBCECloop}
\end{figure}