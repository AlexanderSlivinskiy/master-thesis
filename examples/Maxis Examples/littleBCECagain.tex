\begin{figure}[h]

\centering
\begin{tikzpicture}
%littleBCEC
\drawdummy (init) at (-2,0) {};
\drawcirc (p) at (0,0) {$\mathsf{p}$};
\drawbox (q) at (4,0) {$\mathsf{q}$};
\drawdummy (mid) at (6,0) {};
\drawbox (1) at (8,2) {$\target$ };
\drawcirc (0) at (8,-2)  {$\bot$};

\draw[->] (init) to (p);
\draw[->]  (p) to[bend left] node [midway,anchor=south] {a}(q) ;
\draw[->]  (q) to [bend left] node [midway,anchor=north] {b} (p);
\draw[-] (q) to node [midway,anchor=south] {c} (mid) ;
\draw (mid) -- (0);% node [midway,anchor=north east] {0.5};
\draw (mid) -- (1);%node [midway,anchor=south east] {0.5};
\draw[->]  (0) to[loop above]  node [midway,anchor=south] {e} (0);
\draw[->]  (1) to [loop above] node [midway,anchor=south] {d} (1) ;

\node[draw=red,rectangle,minimum width=6.2cm,minimum height=2.5cm,line width=1mm] (bbox) at (2,0) {};
\node[draw=red,rectangle,minimum width=2cm,minimum height=3.1cm,line width=1mm] (cbox) at (8,-1.4) {};

\end{tikzpicture}
\caption{The same $\SG$ as in Figure \ref{ex:littleBCEC}. The red boxes mark the parts of the state space where the upper bound does not converge, i.e. the controlled end components.}
\label{littleBCECagain}
\end{figure}

