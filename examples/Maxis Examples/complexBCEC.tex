\begin{figure}[h]

\centering
\begin{tikzpicture}
%complex BCEC

\drawbox (A) at (0,4) {p};
\drawbox (B) at (4,4) {q};
\drawcirc (C) at (2,2) {r};
\drawbox (D) at (4,0) {s};
\drawbox (E) at (0,0) {t};

\node (Ao) at (0,6) {0.2};
\node (Bo) at (4,6) {0.4};
\node (Do) at (4,-2) {0.3};
\node (Eo) at (0,-2) {0.1};

\draw[->] (A) to (B);
\draw[->] (B) to (C);
\draw[->] (C) to (D);
\draw[->] (C) to (A);
\draw[->] (D) to (E);
\draw[->] (E) to (C);

\draw[->] (A) to (Ao);
\draw[->] (B) to (Bo);
\draw[->] (D) to (Do);
\draw[->] (E) to (Eo);

\end{tikzpicture}
\caption{An example of a MEC that is also a $\complex~\BCEC$. The numbers on the leaving arrows denote the values of the leaving actions.
%$a$, $b$, $c$, $d$ and $e$ denote the values on the leaving actions.
}
\label{ex:complexBCEC}

\end{figure}