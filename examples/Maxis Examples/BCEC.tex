\begin{figure}[h]

\centering
\begin{tikzpicture}
%BCEC
\drawdummy (init) at (-2,0) {};
\drawcirc (p) at (0,0) {$\mathsf{p}$};
\drawbox (q) at (4,0) {$\mathsf{q}$};


\node (Ao) at (0,-3) {0.8};
\node (Bo) at (4,-3) {0.4};


\draw[->] (init) to (p);

\draw[->]  (p) to[bend left] node [midway,anchor=south] {$\mathsf{a}$}(q) ;
\draw[->]  (q) to [bend left] node [midway,anchor=north] {$\mathsf{c}$} (p);

\draw[->] (p) to node [midway,anchor=east] {$\mathsf{b}$} (Ao);
\draw[->] (q) to node [midway,anchor=east] {$\mathsf{d}$} (Bo);

\node[draw=red,rectangle,minimum width=6.2cm,minimum height=2.8cm,line width=1mm] (cbox) at (2,0) {};


\end{tikzpicture}
\caption{An $\SG$ with an example of a maxmizer-controlled end component in the red box. This example shows why we need the second condition in Definition \ref{def:BCEC}. The numbers on the leaving actions (i.e. $\mathsf{b}$ and $\mathsf{d}$) indicate the value of that action.}
\label{ex:BCEC}
\end{figure}

