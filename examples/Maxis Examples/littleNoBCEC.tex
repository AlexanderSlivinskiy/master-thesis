\begin{figure}[h]
\caption{The same $\SG$ as in Figure \ref{ex:littleBCEC}, but with action $b$ removed}
\label{ex:littleNoBCEC}
\begin{tikzpicture}
%littleqCEC
\drawdummy (init) at (-2,0) {};
\drawcirc (p) at (0,0) {p};
\drawbox (q) at (4,0) {q};
\drawdummy (mid) at (6,0) {};
\drawbox (1) at (8,2) {$\target$ };
\drawcirc (0) at (8,-2)  {$\bot$};

\draw[->] (init) to (p);
\draw[->]  (p) to[bend left] node [midway,anchor=south] {a}(q) ;
\draw[-] (q) to node [midway,anchor=south] {c} (mid) ;
\draw (mid) -- (0);% node [midway,anchor=north east] {0.5};
\draw (mid) -- (1);%node [midway,anchor=south east] {0.5};
\draw[->]  (0) to[loop above]  node [midway,anchor=south] {e} (0);
\draw[->]  (1) to [loop above] node [midway,anchor=south] {d} (1) ;


\end{tikzpicture}
\end{figure}



\iffalse

VI from below in two steps
BVI fails, Illusion
BRTDP below in one simulation if we go to 0, could also never go to 0 and never terminate from below; but this a.s. does not happen; also fails from above

More examples: 
	- VI off by 1 CHECK (red edge smaller than two times epsilon (depends on how term crit is implemented); then two iterations (one to propagate 0.5 to C, one to propagate a very small number to q) => ``no'' change => result 0)
	- littleqCEC with selfloop (takes forever to converge) CHECK
	- Converging EC (Prelim and BVIwo) CHECK
	- Complex BCEC (5 States, without starting state, with a b c d on the exits; also for the things are equal case) CHECK
\fi