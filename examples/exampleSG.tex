\begin{figure}


\centering
\begin{tikzpicture}[scale=0.7, every node/.style={font=\large, transform shape}]
%little BCEC with loop
\drawdummy (init) at (-2,0) {};
\drawcirc (p) at (0,0) {$\mathsf{s_0}$};
\drawbox (q) at (4,0) {$\mathsf{s_1}$};
\drawcirc (v) at (4,-3) {$\mathsf{s_2}$};
\drawdummy (vHelp1) at (4,-4) {};
\drawdummy (vHelp2) at (10,-4) {};
\drawdummy (vHelp3) at (10,2) {};
\drawdummy (mid) at (6,0) {};
\drawbox (1) at (8,2) {$\target$ };
\drawcirc (0) at (8,-2)  {$\mathrm{\zeroSink}$};

\draw[->] (init) to (p);
\draw[->] (p) to node [midway,anchor=south] {$\mathsf{a}$}(q) ;
%\draw[->] (p) to [bend right] node [text width = 0.5cm, below, midway] {$\mathsf{b}$} (v);
\draw[->] (q) to node [text width = 0.75cm, below, midway] {$\mathsf{a}$} (v);
\draw[->] (v) to [bend right] node [text width = 0.5cm, above, midway] {$\mathsf{a}$} (0);
\draw[-] (v) to (vHelp1);
\draw[-] (vHelp1) to node [below, midway] {$\mathsf{b}$} (vHelp2);
\draw[-] (vHelp2) to (vHelp3);
\draw[->] (vHelp3) to (1);
\draw[-] (q) to node [midway,anchor=south] {$\mathsf{b}$} (mid) ;
%\draw (mid) -- (0);% node [midway,anchor=north east] {0.5};
%\draw (mid) -- (1);% node [midway,anchor=south east] {0.5};
\draw[->] (mid) to [bend right] node [text width =1.4cm, align=left, midway, below] {$0.25$} (0);
\draw[->] (mid) to [bend left] node [text width =1.2cm, near end, above] {$0.75$} (1);
%\draw (mid) to [bend left,out=270,in=270] (q) ;
\draw[->]  (0) to[loop above]  node [midway,anchor=south] {$\mathsf{a}$} (0);
\draw[->]  (1) to [loop above] node [midway,anchor=south] {$\mathsf{a}$} (1) ;


\end{tikzpicture}
\caption[Example of a simple stochastic game]{An example of a simple stochastic game. $s_0, s_1, s_2, \target, \mathrm{\zeroSink}$ are states. 
$a,b$ are actions. $s_1$ and $\target$ are Maximizer-states, while the rest of the states belongs to the Minimizer. 
$\target$ is a target and $\mathrm{\zeroSink}$ is a so-called sink, i.e. a state that has no path to the target. 
0.75 and 0.25 are the transition probabilities that a token in $s_1$ moved through action $b$ would either be moved to $\target$ or $\mathrm{\zeroSink}$}
\label{ex:exampleSG}
\end{figure}